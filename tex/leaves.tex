\usemodule[vim]

\input design
\mainlanguage[de]

\starttext

\input front_matter

\section{Einleitung}

Die Phänotypisierung von Strukturen verschiedener Dimensionen, von subzellular bis po\-pu\-la\-tions\-groß, stellt heute eine wesentliche Limitierung bei der Durchführung biologischer Studien dar \cite[kappel].
Zunehmend ist es möglich, in kurzer Zeit enorme Mengen an Rohdaten zu erzeugen. Dieser Trend ist für DNA-Sequenzdaten schon länger zu beobachten, trifft zunehmend aber auch für Bilddaten zu. Die Entwicklung effizienter Verarbeitungstechniken großer Datenmengen hält damit teilweise nicht Schritt.

Für die Analyse von Bilddaten werden daher klar definierte Deskriptoren benötigt, die in der Lage sind, Unterschiede präzise zu beschreiben, wobei die Algorithmen zur Berechnung dieser effizient arbeiten müssen.

Das Ziel dieser Arbeit ist, die in dem Kurs {\emph Bioimage Analysis & Extended Phenotyping} ge\-wonnen\-en Erkenntnisse – Bildverarbeitungstechniken und ihre Verwendung in biologischer Forschung – auf einen realistischen Datensatz anzuwenden und zu versuchen, ein typisches Problem, die Klassifizierung von Strukturen (Blätter nordamerikanischer Baumarten), zu lösen.

\section{Methoden}

\subsection{Ansatz und Implementierung}

Es wurde ein Datensatz analysiert, der jeweils 100–150 Scans von Blättern 185 nordamerikanischer Baumarten enthält \cite[leafsnap].

Zur Beschreibung der Blattformen wurden vier verschiedene morphometrische Deskriptoren verwendet, für welche hier die gängigen englischen Bezeichungen verwendet werden: extent, solidity, eccentricity \& roundness.
Diese wurden ausgewählt, weil sie einfach zu ermitteln sind und außerdem erlauben, fälschlicherweise erkannte Objekte in den Blattscans heraus zu filtern.

Ein weiteres Merkmal dieser morphometrischen Deskriptoren ist, dass sie, auf das jeweilige Objekt bezogen, relative Maße sind. Sie vergleichen allesamt, ein für die Blattform ermittelten Wert mit einem Referenzobjekt, beispielsweise einem Rechteck oder Kreis. Sie nehmen daher Werte zwischen 0 und 1 an. 

Die Verwendung nicht relativer Maße wäre nur möglich, wenn der Maßstab, der jedem Scan beigefügt ist, ebenfalls analysiert und eine Normalisierung der ermittelten absoluten Deskriptoren vorgenommen würde. Für diese Arbeit wurde darauf verzichtet, weswegen nur relative Maße zur Analyse herangezogen wurden.
In den Abschnitten 2.1–2.4 werden die verwendeten Deskriptoren eingehender eingehender beschrieben. 

Die gesamte Analyse wurde in der Programmiersprache {\emph Python}, Version 3.5.2, implementiert. Neben verschiedenen Modulen der Standardbibliothek wurden folgende Bibliotheken verwendet: numpy-10.1.4, pandas-0.18.0, scikit-image-0.12.3, joblib-0.9.4, seaborn-0.7.1.

Um die Rechenzeit zu reduzieren wurde die Berechnung der Metriken für jeden der Blattscans parallel durchgeführt, indem mehrere Python-Prozesse gestartet wurden. Da die durchgeführten Berechnungen für ein einzelnes Bild völlig unabhängig von den Berechnungen für andere waren, handelte es sich hier um ein sogenanntes {\emph embarrassingly parallel} Problem, und die Aufteilung in parallel ausführbare Schritte war dementsprechend mit geringem Aufwand möglich. 

\subsection{Morphometrische Deskriptoren}

\subsubsection{Extent}

Der Extent (Ausdehnung) ist das Verhältnis der Fläche des betrachteten Objekts zur Fläche des dieses Objekt umgebenden Rechtecks {\emph (enclosing oder bounding box).}

\startformula
extent = \frac{object\; area}{enclosing\; box\; area}
\stopformula

Ein Wert von 1 bedeutet, dass das untersuchte Objekt die umfassende Box komplett ausfüllt, also dass das Objekt selbst eine rechteckige Form hat. Das ermöglicht neben der Beschreibung verschiedener Blattformen auch das Entfernen von Objekten, die keine Blätter sein können, da rechteckige Blätter nicht existieren, zumindest nicht im Rahmen dieser Untersuchung. Derartige unerwünschte Objekte sind in der Regel die im Bild vorhandenen Maßstabs- und Farbmarkierungen.

\subsubsection{Solidity}

Die Solidity (ungefähr: Festigkeit) ist ähnlich wie der Extent definiert. Statt der {\emph enclosing box} wird hier allerdings die {\emph convex hull} (konvexe Hülle) als Vergleichsmaß verwendet.

\startformula
solidity = \frac{object\; area}{convex\; hull\; area}
\stopformula

Hier bedeutet ein Wert von 1, dass das untersuchte Objekt die konvexe Hülle ganz ausfüllt. Damit sollten gelappte und gefiederte Formen gut von ganzrandigen oder nur wenig gezahnten Formen unterschieden werden können.

\subsubsection{Eccentricity}

Die Eccentricity (Exzentrizität) beschreibt die Abweichung einer Ellipse von der Kreisform. Ein Wert von 0 bedeutet dass die Ellipse ein Kreis ist, ein Wert von 1, dass die Ellipse eine Linie approximiert. 
Zur Ermittlung dieses Werts wird jene Ellipse betrachtet, die im Bezug auf das untersuchte Objekt, die gleichen zentralen Momente aufweist. Hiermit können längliche Formen gut von gleichmäßig um ein Zentrum ausgedehnten Formen unterschieden werden. Auch dieser Wert ließe sich zum Filtern von länglichen, rechteckigen Objekten verwenden. In dieser Arbeit wurde sich damit jedoch auf den Extent beschränkt. 

\subsubsection{Roundness}

Die Roundness (Rundheit) eines Objekts wird anhand dessen Fläche und Umfang berechnet. Hierbei wird die Ähnlichkeit des Objekts zu einem Kreis bestimmt, indem die Fläche ins Verhältnis zum Umfang gesetzt wird.

\startformula
roundness = \frac{4 \times π \times area}{perimeter^2}
\stopformula

Ein Wert von 1 bedeutet, dass das Objekt ein perfekter Kreis ist, andernfalls ist der Wert kleiner. Die Roundness hat eine ähnliche Anwendbarkeit zur Unterscheidung von Formen, wie die Eccentricity.

\subsection{Segmentierung}

Als Ausgangsmaterial wurden Scans von kompletten Blättern verwendet. Da die Maßstabs- und Farbschlüssel nicht benötigt wurden, wurden eingangs die Bilder so beschnitten, dass in den meisten Fällen diese nicht mehr vorhanden waren. In den nächsten Schritten wurde durch Segmentierung jeweils ein Binärbild berechnet, dass im Idealfall nur noch ein weißes Objekt, das Blatt auf schwarzem Hintergrund, zeigte. 

Das bedurfte der Umwandlung der Farbbilder (RGB) in Graustufenbilder. Hierzu wurde keine Umwandlung durch Gewichtung der Farbkanäle vorgenommen, sondern nur der Grünanteil extrahiert. Damit war bereits die Umwandlung in Graustufen gegeben. Diesem Ansatz wurde der Vorzug gegeben, weil Blätter in aller Regel einen hohen Grünanteil und nur geringe Rot- und Blauanteile aufweisen. Die Idee dahinter war, den Einfluss störender Objekte, die hohe Rot- und Blauanteile haben, zu reduzieren, bevor das Thresholding vorgenommen wurde. Üblicherweise wird bei der Umwandlung in Graustufen ohnehin eine Gewichtung zugunsten des Grünanteils vorgenommen, so dass diese Methode dahingehend nur weiter geht.

Im Folgenden wurde die eigentliche Segmentierung der vorhandenen Objekte mittels globalen Thresholdings nach Otsu vorgenommen. Die so erhaltenen Binärbilder enthielten oft nur noch ein Objekt: das Blatt. Um die am häufigsten identifizierten Fehler zu minimieren wurden noch weitere Schritte unternommen. Es wurde {\emph closing} der Strukturen vorgenommen, um kleine Löcher in größeren Strukturen zu schließen. Des Weiteren wurden Objekte mit einer Fläche kleiner als 128 Pixel entfernt.

Im nächsten Schritt wurden, nach der Feststellung separater Regionen {\emph (labelling),} die Eigenschaften aller verbleibenden Objekte berechnet. Da teilweise immernoch Maßstabs- und Farbschlüssel, die nicht durch das Beschneiden der Bilder entfernt wurden, vorhanden waren, wurden in einem letzten Schritt vor der Aufzeichnung der Daten, bestimmte Objekte entfernt. Objekte mit einem Extent größer als 0.8, die mit hoher Wahrscheinlichkeit keine Blätter oder Teile dieser repräsentierten, wurden für von der Analyse ausgeschlossen.

\subsection{Klassifikation}

Für jedes Blatt jeder Art in dem Datensatz wurden die Deskriptoren berechnet. Für jede Art wurden daraus jeweils die Mediane berechnet. Eine Art wurde somit durch die Mediane der vier Deskriptoren repräsentiert. Der Median wurde verwendet, um den Einfluss von Extremwerten zu reduzieren und um damit gleichzeitig eine letzte Fehlerminimierung vorzunehmen.

Der Grad der Ähnlichkeit eines Blatts mit den verschiedenen Baumarten wurde festgestellt, indem die euklidische Distanz der Deskriptorwerte des einzelnen, zuzuordnenden Blatts mit den Medianen der Deskriptorwerte der ausgewählten Arten verglichen wurde.

Dieser Vergleich wurde vorgenommen, indem die beschreibenden Deskriptoren als Koordinaten im euklidischen Raum betrachet wurden. Die euklidische Distanz ist dann wie folgt definiert:

\startformula
d(x, y) = \sqrt{(x_1 - y_1)^2 + ... + (x_n - y_n)^2}
\stopformula
 
Hierbei sind $x$ und $y$ Punkte, die durch ihre Koordinaten gegeben sind: $x = (x_1,...,x_n)$ und $y = (y_1,...,y_n)$.

In diesem Fall, mit durch vier Koordinaten, den Deskriptorwerten, gegebenen Punkten lautet die Gleichung zur Berechnung der Distanz:

\startformula
d(M,B) = \sqrt{ (ext_M - ext_B)^2 + (sol_M - sol_B)^2 + (ecc_M - ecc_B)^2 + (rnd_M - rnd_B)^2 }
\stopformula

mit $M$ ... Median der Art, $B$ ... Wert des Deskriptors des zuzuordnenden Blatts, $ext$ ... Extent, $sol$ ... Solidity, $ecc$ ... Eccentricity, $rnd$ ... Roundness.

Je kleiner die Distanz, desto wahrscheinlicher ist es, dass ein Blatt einer bestimmten Art zuzuordnen ist. Eine definitive Zuordnung ist also durch das Auswählen der Art möglich, zu der die geringste Distanz besteht. Da alle vier Deskriptoren Werte zwischen 0 und 1 annehmen, nimmt $d(M,B)$ Werte zwischen 0 und 2 an. In den Ergebnissen wurde die Distanz des Verständnisses halber durch zwei geteilt, damit wieder Werte zwischen 0 und 1 erreicht werden.

Die Arten für die Analyse wurde so gewählt, dass einige der gleichen Gattung angehörten und sich recht ähnlich sahen und die anderen ansonsten sehr verschiedene Formen aufwiesen. Um einen Eindruck von der Funktionsfähigkeit der Analyse zu bekommen, wurde für jede Art versucht, eines der Blätter zuzuordnen, dass auch bei bei der Berechnung der Mediane mit eingeflossen ist. Dieses wurde willkürlich ausgewählt.

Des Weiteren wurden bereits segmentierte Bilddaten, die im Feld fotografiert und nicht im Labor gescannt wurden, verwendet, um die Güte des Algorithmus zu testen. Für diese bereits vorliegenden Binärbilder wurden auf identische Art, wie für die zuvor analysierten Bilder die Deskriptoren bestimmt. Für jedes Blatt wurde anhand dessen Metriken dann eine Zuordnung zu den Arten vorgenommen und zusammengefasst, wie oft die Zuordnung korrekt war.

\section{Ergebnisse}

Die Zuordnung eines willkürlich ausgewählten, einzelnen Blattes zu den Arten funktionierte in 8 von 10 Fällen (Abb. 1). Hier war allerdings das Blatt aus der Menge entnommen, aus der auch die Mediane für die Zuordnung berechnet wurden. Dennoch zeigt sich, dass prinzipiell mit diesem Algorithmus gearbeitet werden kann.

Zwei mal schlug die Zuordnung fehl, wobei das Blatt sogar einer anderen Gattung zugeordnet wurden (Abb. 1). Auffallend ist, dass häufig die Zuordnung zwar korrekt, der Abstand zu einer anderen Art aber durchaus sehr gering war, so dass eine falsche Zuordnung nahe lag (vgl. {\emph A. palmatum, A. saccharinum, J. nigra, R. pseudoacacia}, Abb. 1).

Am eindeutigsten war die Zuordnung von {\emph A. arborea}, obwohl hier die Distanz zur eigenen, korrekten Art größer ist, als für andere Blätter. Allerdings sind die Distanzen zu den inkorrekten Arten durchweg um einiges größer, als das bei anderen Blättern der Fall ist (Abb. 1). 

\placefigure[force]{Darstellung der Distanzen zwischen den Punkten, die durch die Mediane der Deskriptorwerte einer Art und den Einzelwerten eines zugeordneten Blattes gegeben sind. Die Werte liegen zwischen 0 und 1. Das Diagramm wird folgendermaßen gelesen: ein Blatt (theoretisch unbekannter Art) auf der y-Achse wird derjenigen Art auf der x-Achse zugeordnet, zu der Distanz am geringsten ist (Vergleiche Werte in einer Reihe). Grün umrandet ist die theoretisch korrekte, rot umrandet ist die nach dem Algorithmus erfolgte Zuordnung.}
{\externalfigure[../out/distances_heat_annotated.pdf][width=\textwidth]} {}

Bei dem eigentlichen Gütetest für die in dieser Arbeit verwendeten Methoden zeigen sich sehr unterschiedliche Ergebnisse. Zum einen wurden verschieden viele Blätter, die im Feld fotografiert wurden, so gut segmentiert und analysiert, dass sie in das Ergebnis einfließen konnten und zum anderen sind die daraufhin korrekt erfolgten Zuordnungen unterschiedlich ausgefallen (Tab. 1).

Für {\emph A. saccharinum, A. glabra, A. arborea, B. populifolia} und {\emph Z. serrata} wurde jedoch in 50\% der Fälle oder mehr eine korrekte Zuordnung erreicht. Allerdings muss bemerkt werden, dass für {\emph A. glabra} so wenige Zuordnungen durchgeführt werden, dass eine Bewertung des Zuordnungserfolgs hier nicht möglich ist. Ebenso lassen sich zu {\emph J. nigra} keine Aussagen treffen. Für {\emph R. pseudoacacia} ist die Analyse fehlgeschlagen und es liegen keine Ergebnisse vor (Tab. 1).

Besonders geringen Erfolg hatte die Zuordnung zu {\emph Acer palmatum}, wohingegen die Zuordnung zu {\emph Acer saccharinum} die höchste Erfolgsquote aufweist. Da beide Arten der gleichen Gattung angehören ist dies besonders auffallend.

\setupTABLE[c][each][align={middle,lohi},frame=off]
\setupTABLE[c][1][width=50mm]
\setupTABLE[c][2,3,4][width=35mm]
\setupTABLE[r][1][topframe=on, rulethickness=1pt, offset=1mm]
\placetable[force]{Korrekte Zuordnungen zur jeweiligen Art in absoluten und prozentualen Werten}
{\bTABLE

\bTR \bTD \bf Art                          \eTD \bTD \bf untersucht \eTD \bTD \bf korrekt \eTD \bTD \bf \% korrekt \eTD \eTR
\bTR[topframe=on, rulethickness=0.5pt, offset=1mm] 
     \bTD \it Abies concolor               \eTD \bTD 41             \eTD \bTD 7           \eTD \bTD 17             \eTD \eTR
\bTR \bTD \it Acer palmatum                \eTD \bTD 30             \eTD \bTD 1           \eTD \bTD 3              \eTD \eTR
\bTR \bTD \it Acer saccharinum             \eTD \bTD 28             \eTD \bTD 20          \eTD \bTD 71             \eTD \eTR
\bTR \bTD \it Aesculus glabra              \eTD \bTD 7              \eTD \bTD 4           \eTD \bTD 57             \eTD \eTR
\bTR \bTD \it Amelanchier arborea          \eTD \bTD 29             \eTD \bTD 19          \eTD \bTD 66             \eTD \eTR
\bTR \bTD \it Betula populifolia           \eTD \bTD 18             \eTD \bTD 9           \eTD \bTD 50             \eTD \eTR
\bTR \bTD \it Juglans nigra                \eTD \bTD 4              \eTD \bTD 0           \eTD \bTD 0              \eTD \eTR
\bTR \bTD \it Metasequoia glyptostroboides \eTD \bTD 40             \eTD \bTD 11          \eTD \bTD 28             \eTD \eTR
\bTR \bTD \it Robinia pseudoacacia         \eTD \bTD NaN            \eTD \bTD NaN         \eTD \bTD NaN            \eTD \eTR
\bTR[bottomframe=on, rulethickness=1pt, offset=1mm]
     \bTD \it Zelkova serrata              \eTD \bTD 61             \eTD \bTD 38          \eTD \bTD 62             \eTD \eTR
\eTABLE}

\section{Diskussion}

Insgesamt kann die hier untersuchte Methode als funktionierend aber fragil bezeichnet werden. An vielen Stellen wurde korrekt zugeordnet, eine falsche Zuordnung war jedoch häufig in Reichweite, insbesondere, weil hier aus vier Dimensionen ein dimensionsloses Maß (Distanz) bestimmt und verwendet wurde. Um das zu erreichen, wurden die Informationen, die in der Verteilung, der zu einer Art gehörenden Referenzpunkte stecken, stark zusammengefasst (Median). Damit gehen Informationen verloren, beispielsweise in welcher »räumlichen« Beziehung ein Punkt (das zuzuordnende Blatt) zu anderen anderen Punkten (der ursprünglichen Punktwolke einer ganzen Art) steht.

Die auffällige Diskrepanz beim Erfolg der Zuordnung von {\emph Acer palmatum} und {\emph Acer saccharinum} lässt sich damit erklären, dass die Blattformen sehr ähnlich sind. Hier ist es vorstellbar, dass vor allem Eccentricity und Solidity kaum Unterschiede aufweisen. Da beide Arten der gleichen Gattung angehören ist es naheliegend, anzunehmen, dass {\emph A. palmatum} häufig {\emph A. saccharinum} zugeordnet wurde. Warum dies nicht in ausgleichendem Umfang auch andersherum gilt, ist jedoch unklar und es muss eine systematische Fehlerquelle bei der Zuordnung ähnlicher Blattformen vermutet werden.

Aber es gibt auch systematische Fehlerquellen in der Bearbeitung der Daten selbst, noch bevor eine Methode zur tatsächlichen Klassifizierung angewendet wird. So ist es häufig der Fall, dass Blätter schwer zu segmentieren sind. Insbesondere Nadelblätter {\emph (Abies concolor)} bereiteten hier Schwierigkeiten.

Des Weiteren wurden nur relative Maße herangezogen und die Anzahl an Deskriptoren war insgesamt sehr gering. Mehr Deskriptoren zu verwenden, würde die Chance erhöhen, dass verschiedene Arten in diesen zusätzlichen Dimensionen Unterschiede aufweisen und somit besser in die korrekten Klassen eingeordnet werden können.

Die verwendete Methode zur Klassifizierung ist sehr simpel, genauso, wie die Verwendung von nur vier morphometrischen Deskriptoren. Hier könnte vermutlich viel erreicht werden, wenn mehr Metriken, beispielsweise Fourier-Analyse oder {\emph landmark-}basierte Analyse der Form, sowie anspruchsvollere Klassifizierungsmethoden verwendet würden.

\page
\subject{Referenzen}

\placepublications[criterium=all]

\subject{Anhang}

\input code_listings

\stoptext
